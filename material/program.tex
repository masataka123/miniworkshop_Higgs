\documentclass[dvipdfmx,a4paper,12pt]{article}
\usepackage[utf8]{inputenc}
%\usepackage[dvipdfmx]{hyperref} %リンクを有効にする
\usepackage{url} %同上
\usepackage{amsmath,amssymb} %もちろん
\usepackage{amsfonts,amsthm,mathtools} %もちろん
\usepackage{braket,physics} %あると便利なやつ
\usepackage{bm} %ラプラシアンで使った
\usepackage[top=20truemm,bottom=30truemm,left=22truemm,right=22truemm]{geometry} %余白設定
\usepackage{latexsym} %ごくたまに必要になる
\renewcommand{\kanjifamilydefault}{\gtdefault}
\usepackage{otf} %宗教上の理由でmin10が嫌いなので


\usepackage[all]{xy}
\usepackage{amsthm,amsmath,amssymb,comment}
\usepackage{amsmath}    % \UTF{00E6}\UTF{0095}°\UTF{00E5}\UTF{00AD}\UTF{00A6}\UTF{00E7}\UTF{0094}¨
\usepackage{amssymb}  
\usepackage{color}
\usepackage{amscd}
\usepackage{amsthm}  
\usepackage{wrapfig}
\usepackage{comment}	
\usepackage{graphicx}
\usepackage{setspace}
\usepackage{pxrubrica}
\usepackage{enumitem}
\usepackage{mathrsfs} 
\usepackage[dvipdfmx]{hyperref}
\setstretch{1.2}

\newcommand{\mathsym}[1]{{}}
\newcommand{\unicode}[1]{{}}

\newcounter{mathematicapage}


%%%%%%%%% Theorem-like environment %%%%%%%%%%%
%
\theoremstyle{plain} %text of this environment is typesetted in italics
\newtheorem{theorem}{\indent\sc Theorem}[section]
\newtheorem{lemma}[theorem]{\indent\sc Lemma}
\newtheorem{corollary}[theorem]{\indent\sc Corollary}
\newtheorem{proposition}[theorem]{\indent\sc Proposition}
\newtheorem{claim}[theorem]{\indent\sc Claim}
\newtheorem{conjecture}[theorem]{\indent\sc Conjecture}
%
\theoremstyle{definition} %text of this environment is typesetted in roman letters
\newtheorem{definition}[theorem]{\indent\sc Definition}
\newtheorem{remark}[theorem]{\indent\sc Remark}
\newtheorem{example}[theorem]{\indent\sc Example}
\newtheorem{notation}[theorem]{\indent\sc Notation}
\newtheorem{assertion}[theorem]{\indent\sc Assertion}
\newtheorem{observation}[theorem]{\indent\sc Observation}
\newtheorem{problem}[theorem]{\indent\sc Problem}
\newtheorem{question}[theorem]{\indent\sc Question}
%
%If a theorem-like environment should not be numbered,
%add * after \newtheorem, and delete the counter option such as [theorem].
\newtheorem*{remark0}{\indent\sc Remark}
%
%%%%% Proof %%%%%
\renewcommand{\proofname}{\indent\sc Proof.}
%The following commands are available in the proof environment:
%\begin{proof}
%\end{proof}
%The end of a proof is marked with a square.
%%%%%%%%%%%%%%%%%%%%%%%%%%%%%%%%%%%%%%%%%
\pagestyle{empty}

\begin{document}

\begin{center}
  {\Huge Mini-workshop on Higgs bundles}
 
  %{\Large -Hodge theory and vanishing theorem-}

  \end{center}
  
\vskip5mm
\begin{flushleft}
{ Date: 27th-28th May 2024. (2024年5月27--28日)}


{Place: Lecture Room E404 in Graduate School of Science Building E in Osaka University (Toyonaka Campus).}
{(大阪大学理学部E404講義室(豊中キャンパス))}

\end{flushleft}


%\footnote{ホームページ: \texttt{https://sites.google.com/site/hisashikasuyamath/workshop-on-complex-geometry-in-osaka-2023?authuser=0}}
%\footnote{This conference is supported by Osaka City University Advanced Mathematical Institute: MEXT Joint Usage/Research Center on Mathematics and Theoretical Physics.}


\vskip3mm
\noindent{\Large \bf Program}

\vskip3mm
\noindent{\bf \underline{27th May (Monday)}}
\vskip1mm
\noindent {\bf 10:00--11:00}
{\bf Laura Schaposnik (University of Illinois)}\\
\hspace{11pt} An introduction to Higgs bundles and their integrable system I
%\vskip3mm

\noindent {\bf 11:30--12:30} 
{\bf Laura Schaposnik (University of Illinois)}\\
\hspace{11pt} An introduction to Higgs bundles and their integrable system II
%\vskip3mm

\noindent {\bf14:30--15:30} 
{\bf Natsuo Miyatake (Mathematical Science Center for Co-creative Society, Tohoku University)}\\
\hspace{11pt} Harmonic metrics on cyclic Higgs bundles, subharmonic functions, and entropy
%\vskip3mm

\noindent {\bf 16:00--17:00} 
{\bf Mengxue Yang (Kavli IPMU, The University of Tokyo)}\\
\hspace{11pt} Conformal limit on Cayley components

%\vskip3mm
\noindent{\bf \underline{28th May (Tuesday)}}
\vskip1mm
\noindent {\bf 10:00--11:00} 
{\bf Laura Schaposnik (University of Illinois)}\\
\hspace{11pt} An introduction to Higgs bundles and their integrable system III



%%%%%%%%%%%%%%%%%%%%%%%%%%%%%%%%%%%%
\begin{comment}

\begin{table}[htb]
\centering
 % \caption{スペック比較:罫線あり}
  \begin{tabular}{| c | | c | c | c |}  \hline
  Time  & 3/22(水) & 3/23(木)& 3/23(金)  \\ \hline 
     \begin{tabular}{c} GMT 6:00-7:00 \\ (JST 15:00-16:00)\end{tabular}
&Shin-ichi Matsumura & Jihun Yum&Seungjae Lee \\ \hline
      \begin{tabular}{c}   GMT 7:10-8:10 \\ (JST 16:10-17:10)  \end{tabular}
& Junchao Shentu & Hoseob Seo&\sout{ Juanyong Wang}\\ \hline
 \begin{tabular}{c}   GMT 8:40-9:40\\ (JST 17:40-18:40) \end{tabular}
&  Feng Hao& F\'elix Lequen& Lukas Braun \\ \hline
 \begin{tabular}{c}   GMT 11:30-12:30\\ (JST 20:30-21:30) \end{tabular}
&  Xiaojun Wu &  Olivier Thom &    \\ \hline
  \end{tabular}
\end{table}
\end{comment}
%%%%%%%%%%%%%%%%%%%%%%%%%%%%%%%%%

\vskip8mm
\noindent{\large \bf Information}

This workshop will be held as a pre-seminar for our conference 
"New developments in Kobayashi-Hitchin correspondence and Higgs bundles"
from 5th-9th August 2024 in Osaka Metropolitan University.

\vskip8mm
\noindent{\large \bf Organizers} \vspace{-11pt}
\begin{itemize}
  \setlength{\parskip}{0cm} 
  \setlength{\itemsep}{0cm}
\item Yoshinori Hashimoto (Osaka Metropolitan University)
\item Masataka Iwai (Osaka University)
\item Hisashi Kasuya (Osaka University)
\item Natsuo Miyatake (Mathematical Science Center for Co-creative Society, Tohoku University,)
  \end{itemize}

%\noindent{\large \bf Supports 1}

%This conference is supported by “Osaka Central Advanced Mathematical Institute (MEXT Promotion of Distinctive Joint Research Center Program JPMXP0723833165), Osaka Metropolitan University”.

\noindent{\large \bf Supports}  \vspace{-11pt}
\begin{itemize}
  \setlength{\parskip}{0cm} 
  \setlength{\itemsep}{0cm}
\item JSPS KAKENHI 19H01787 Grant-in-Aid for Scientific Research (B)
%\item JSPS KAKENHI 20K03592 Grant-in-Aid for Scientific Research (C)
%\item JSPS KAKENHI 22K13907 Grant-in-Aid for Early Career Scientists.
%\item JSPS KAKENHI 23K03120 Grant-in-Aid for Scientific Research (C)
\item JSPS KAKENHI 24K16912 Grant-in-Aid for Early Career Scientists.  
\end{itemize}

%\noindent{\large \bf Homepage}

%We have posted various information on our website, including how to access to the conference room "Lecture Room E404".

%\vskip3mm
%Homepage Link: \url{https://masataka123.github.io/miniworkshop_Higgs/}

%You can also read the QR code below:

%\begin{figure}[htbp]
%\begin{center}
% \includegraphics[height=50mm, width=50mm]{Higgs.png}
%\end{center}
%\end{figure}




\newpage

\noindent{\Large \bf Abstract}
\vskip5mm

%\noindent{\bf 27th May (Monday)}
%\vskip3mm

\noindent {\bf Laura Schaposnik (University of Illinois)}\\
An introduction to Higgs bundles and their integrable system.

\vskip1mm
During the mini-course we will introduce Higgs bundles and
their integrable system by first considering the basic definitions,
and slowly introducing the Hitchin fibration. We will then look at the
Hitchin fibration for different groups and see how dualities arise (be
them from mirror symmetry, or from other correspondences such as low
rank isogenies). Finally, we will dedicate the last talk to the
introduction of some particular Lagrangians in the moduli space of
Higgs bundles giving branes, whose understanding can lead to insights
in representation theory ( about representations of 3-manifolds,
equivariant representations, etc).
\vskip5mm


\noindent {\bf Natsuo Miyatake (Mathematical Science Center for Co-creative Society, Tohoku University)}\\
Harmonic metrics on cyclic Higgs bundles, subharmonic functions, and entropy

\vskip1mm
Let $X$ be a Riemann surface and $K_X \rightarrow X$ the canonical bundle. For each integer $r \geq 2$, each $q \in H^0(K_X^r)$, and each choice of the square root $K_X^{1/2}$ of the canonical bundle, we canonically obtain a Higgs bundle, which is called a cyclic Higgs bundle. In this talk, I will introduce several new notions regarding cyclic Higgs bundles. First, I will introduce the notion of cyclic Higgs bundles with multi-valued Higgs fields and their associated Hitchin equation. Second, I will introduce a generalization of the Hitchin equation for cyclic Higgs bundles associated with a quasi-subharmonic function, obtained by infinitely increasing the degree of multivalence of the multi-valued Higgs field. Third, I will further generalize this generalized Hitchin equation to complex higher-dimensional manifolds. Finally, I will propose a new concept, which I call cyclic entropy, defined using the solution to the generalized Hitchin equation for cyclic Higgs bundles associated with a quasi-subharmonic function. One of the motivations for introducing these new notions comes from weighted potential theory, including the theory of the asymptotic behavior of sections of holomorphic line bundles. I will present the results obtained so far regarding these new concepts and discuss their potential for further development.
\vskip5mm

\noindent {\bf Mengxue Yang (Kavli IPMU, The University of Tokyo)}\\
Conformal limit on Cayley components

\vskip1mm
In 2014, Gaiotto conjectured that there is a biholomorphism between Hitchin components and spaces of opers on a punctured sphere via a scaling limit called the $\hbar$-conformal limit. On a compact Riemann surface of $g \ge 2$, this biholomorphism has been proven in 2016. Motivated by the study of higher Teichm\"uller spaces, we may view the Hitchin components as a part of a larger family of special components called Cayley components. I will talk about the Cayley components and propose their conformal limit to be the generalized notion of opers of Collier—Sanders.
%\vskip4mm


\end{document}